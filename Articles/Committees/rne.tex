\item The Rules and Elections Committee:
\begin{enumerate}
\item Consists of a maximum of ten members:
\begin{enumerate}
\item The Chair, one representative from each of the Graduate Council, Undergraduate Council, Executive Board, Judicial
Board, Interfraternity Council, Panhellenic Council, one Independent\footnote{Independent shall be defined as any student not affiliated with a Greek organization} member appointed by the Chair and approved
by a majority vote of the committee, one member-at-large appointed by the Chair and approved by a majority vote of
the committee, and one member-at-large from the voting membership of the Student Senate appointed by the Grand
Marshal.
\item Each of these groups shall appoint its own representative by the end of the second full week of the fall semester.
Should any of the memberships, with the exception of the chair, not be appointed by the required time or be vacant
for more than three weeks, the chair shall appoint a member within two weeks of the body’s failure to do so matching the
eligibility criteria of the group normally responsible for making the appointment, to be confirmed by majority vote of the
Senate. The group responsible for appointing a person to that membership shall forfeit its right to do so until the following
semester.
\item All members must be Activity Fee paying students. All members, with the exception of the chair, shall serve through
the end of the spring semester, until replaced, or until resigned, whichever comes first. The chair shall serve until they
resign or are replaced.
\item Quorum for the Rules and Elections Committee shall be $\frac{2}{3}$ of the total membership rounded up to the nearest
person. The Rules and Elections Committee shall not meet unless a Senator is appointed to the committee.
\item A member of the Rules and Elections Committee who is a candidate for Grand Marshal, President of the Union or
joins a political party or is a candidate assistant, as defined by the Student Senate or the Rules and Elections
Committee, will automatically be removed from the Rules and Elections Committee.
\end{enumerate}

\item Its duties shall include the following:
\begin{enumerate}
\item The Committee shall be responsible for reviewing and approving the language of all constitutional amendments that
come before the Senate for approval.
\item It shall be responsible for properly publicizing, conducting and supervising all elections for officers and
representatives of the governing bodies of the Union. It shall submit a report of the election results to the Student
Senate following the election.
\item It shall, at the discretion of the Senate, establish rules for, supervise, and investigate any student government election or referendum for the Rensselaer Union.
\item It shall prepare and submit to the Judicial Board and Senate reports on all contested elections, which it investigates.
\end{enumerate}

\item The Chair shall appoint a vice chair from the voting membership of the committee. Duties of the Vice Chair shall include
assisting the Chair in the day-to-day administration of the committee. In the event that the Chair is absent or unable to run a
meeting the Vice Chair shall have the authority to chair a meeting. At such a meeting, quorum shall be $\frac{2}{3}$ the voting
membership, less the Vice Chair. The Vice Chair shall not have a vote at any meeting that they chair. In the event of the
resignation, removal, or extended absence of the Chair, the Vice Chair shall assume the duties of the Chair, until the Chair
is able to resume their duties, or until a new chair has been appointed.
\end{enumerate}
