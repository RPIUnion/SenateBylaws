\article{Legislation}

\begin{enumerate}

    \item \textbf{Legislation may be introduced for consideration by the Senate in any meeting through the following ways:
    \begin{enumerate}
        \item As a proposal, report, or recommendation of a standing committee or subcommittee.
        \item As a proposal from the Graduate Council or the Undergraduate Council.
        \item As a proposal sponsored by two or more Senators.
    \end{enumerate}}

    \item All proposed legislation must be submitted to the Grand Marshal who will, when appropriate, assign the legislation to the applicable committee for study and recommendation. In some cases this will occur on the Senate floor to ensure proper consideration within committee first and to allow business to proceed.

    \item \textbf{Legislation shall be submitted to the Grand Marshal no later than three days prior to the meeting during which it is to be motioned and discussed. Legislation that is proposed by two or more Senators may be submitted and discussed within the same meeting.}

    \item \textbf{The Bylaws may be temporarily suspended by a two-thirds vote. This suspension will stand until any of the following occur:
    \begin{enumerate}
        \item The meeting in which the Bylaws were temporarily suspended adjourned; or
        \item The Senate reinstates these Bylaws by majority vote.
    \end{enumerate}
    }

    \item The Student Senate may not unilaterally repeal a bylaw of another body.

    \item \textbf{Where not provided for in the \textit{Rensselaer Union Constitution}, the Senate shall require a majority vote of to approve a constitution or set of bylaws submitted to it by another body.}

    \item The Student Senate shall keep a record of all legislation that comes into consideration. To ensure accurate record keeping and to provide context for posterity, the Senate shall include any supplementary materials that are relevant to the recorded legislation.

\end{enumerate}
