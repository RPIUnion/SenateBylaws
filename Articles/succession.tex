\article{Succession}
\begin{enumerate}
    \item If the position of the Grand Marshal is vacant, or if the Grand Marshal is unable to fulfill the duties or requirements of office, then the Vice Grand Marshal shall assume the duties of the Grand Marshal as Acting Chair of the Senate.

    \item The Acting Chair of the Senate assumes the duties of Grand Marshal until such a time that a meeting of the Senate can be
    called. The first and only order of business of such a meeting shall be the selection of a new Grand Marshal.

    \item All nominations for the position of Grand Marshal must be motioned and seconded by the members of the Senate. The
    nomination processes shall be overseen by the Chair of the Judicial Board.

    \item In the case of multiple nominees, an appointment for Grand Marshal shall be selected by a series of simple majority votes.
    \begin{enumerate}
        \item If there are two nominees, a vote shall be taken, and the nominee with a simple majority shall be named the appointment
        for Grand Marshal.

        \item If there are three or more nominees, a series of votes shall be taken.
        \begin{enumerate}
            \item In each vote, the nominee with the least number of votes shall be removed from consideration, and only the remaining
            nominees shall be considered in the next vote.

            \item This series of votes shall continue until a single nominee remains, and that nominee shall be named the appointment for
            Grand Marshal.
        \end{enumerate}
    \end{enumerate}

    \item The nominee who is named the appointment for Grand Marshal must be approved by $\frac{2}{3}$ of the Senate’s total voting membership. If the appointment is not approved, a new process of nominations begins.
    \item Once the appointment of a new Grand Marshal has been approved by the Senate, the Acting Chair of the Senate shall immediately relinquish the responsibilities of office to the new Grand Marshal.

\end{enumerate}
