\article{Petitions}
\begin{enumerate}
    \item Petitions shall be a means by which members of the Union may bring concerns, initiatives, or ideas before the Student Senate.
    \begin{enumerate}
        \item Petitions may only be created and signed by members of the Union.
        \begin{enumerate}
            \item Should a signatory lose their membership status during the duration of a petition they have signed, their signature on that
            petition shall remain valid.
        \end{enumerate}

        \item The Rules and Elections Committee shall maintain rules governing the petition process.
        \begin{enumerate}
            \item Such rules shall be approved by a majority vote of the Senate.
        \end{enumerate}
    \end{enumerate}

    \item Should any petition, compliant with rules governing the petition process, reach a threshold of 250 signatures within one full
    calendar year of its posting, it shall be placed on the agenda at a Senate meeting occurring within the next fifteen academic
    days, excluding exam periods.
    \begin{enumerate}
        \item The Senate may choose to address petitions at any time before this.

        \item Should the Senate’s term end within fifteen academic days of a petition reaching this threshold, that petition may instead be
        addressed by the next Senate in the semester of their election.
    \end{enumerate}

    \item The Senate may choose to take the following courses of action when addressing a petition:
    \begin{enumerate}
        \item Charge one or more Senators to investigate issues raised by the petition.

        \item Charge a committee to investigate the petition as an initiative.

        \item Vote on a resolution on the issue.

        \item Vote to hold a referendum election on the issue.

        \item Refer a petition to another organization as appropriate.
    \end{enumerate}
\end{enumerate}
