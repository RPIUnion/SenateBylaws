\article{Rules for Voting and Debate}

\begin{enumerate}

\item Upon entering debate, a queue will be established and maintained by the Parliamentarian.
\begin{enumerate}
\item The queue determines who has the floor to speak. The Parliamentarian is responsible for acknowledging all those who wish
to speak, and must announce each speaker before their speaking time begins.
\item The queue’s topic defers to the discussion at hand.
\item Total debate time is unbounded by default.
\item Speaking time defaults to five minutes.
\item Anyone present in the room has the right to speak, and may speak any number of times.
\begin{enumerate}

\item Those who have not spoken will be called upon before those who have spoken.

\item If the queue is on the topic of a presentation, designated presenters may respond to any questions directed at them.
Speaking time limits still apply.

\item The Parliamentarian may call any speaker off-topic or out of time.

\end{enumerate}\item Queues for any motion will proceed until the motion has left the floor. Other queues may be ended by the presiding officer
when empty.
\item A Senator may move to restrict the queue, requiring a second and majority approval.
\begin{enumerate}
\item By default, only members of the Senate, Officers, and Committee Chairs may be added to a restricted queue.
Additional individuals to be given queue rights may be specified by the Senator moving to restrict the queue.
\item If an individual without queue rights wishes to speak during a restricted queue, they will not be called upon until no
members with queue rights remain on the queue.
\end{enumerate}

\item A Senator may move to close the queue, requiring a second and a $\frac{2}{3}$ vote.
\begin{enumerate}
\item No additional individuals may be added to a closed queue.
\end{enumerate}

\item A Senator may move to reopen a closed or restricted queue, requiring a second and a majority vote.
\item When speaking time is limited, speakers can yield their time in the following ways:
\begin{enumerate}
\item Yield to the Chair, forfeiting their time.
\item Yield to another speaker, transferring their time to another person in the room. That person may decline to speak,
forfeiting their time.
\item A yielded speaker may not yield time further.
\end{enumerate}

\item If unfriendly amendments or subsidiary motions requiring debate are made, the effective queue will be suspended and the
Parliamentarian will open a new queue.
\begin{enumerate}
\item When the amendment or subsidiary motion leaves the floor, its corresponding queue will be ended, and the original
queue will be resumed.
\end{enumerate}
\end{enumerate}

\item Senators are the only members of the Student Senate that have the right to vote on any business of the Senate.
\begin{enumerate}
\item The presiding officer may only vote to break a tie, unless otherwise specified within the Union Constitution, these Bylaws,
or Robert’s Rules of Order. The presiding officer is permitted to abstain. If a tie remains, the motion fails.
\end{enumerate}

\item All voting shall be by a show of hands, unless a Senator, Officer, Committee Chair, or the presiding officer requests a roll call.
\begin{enumerate}
\item In a show of hands, the presiding officer will first request the number in favor, then the number opposed. The presiding
officer will then announce the vote, and whether the motion has passed. In the event of a tie, the presiding officer may then
announce their vote followed by the outcome.
\item If a roll call vote is requested, the Secretary will read the names of the Senators alphabetically by surname. If the Grand
Marshal is included in the voting population, he or she will be called last.
\item Each Senator may vote “yes”, “no”, or “pass”. If they pass, their name will be read again after the first completion of the
roll call. At this time, the Senator must vote or abstain.
\item No vote may be determined, nor any result announced, until all Senators present have voted or abstained.
\end{enumerate}

\item All legislation considered by the Student Senate requires majority approval unless mentioned otherwise in these Bylaws, or in
Robert’s Rules of Order. All votes, unless specified within these Bylaws or in Robert’s Rules of Order, are debatable.
\end{enumerate}