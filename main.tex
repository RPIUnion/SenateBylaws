\documentclass{bylaws}
\usepackage[utf8]{inputenc}

\title{The Bylaws of the Rensselaer Union Student Senate}
\author{As revised April 2, 2015}
\date{}

\begin{document}

\maketitle

\article{Organization}
\begin{enumerate}
\item The name of this organization shall be the Rensselaer Union Student Senate, hereinafter referred to as the Student Senate or the
Senate.
\item The Student Senate gains its authority from the Rensselaer Union Constitution.
\item The Student Senate shall pass legislation commensurate with its duties as defined by the Rensselaer Union Constitution, and as
specified in these Bylaws.
\end{enumerate}


\article{Membership}
\begin{enumerate}
\item A Senator shall be defined as any person who has voting rights in the Senate, as defined by the Rensselaer Union Constitution.
Senate membership shall consist of the Grand Marshal, and Senators as defined by the Rensselaer Union Constitution.
\item The voting membership of the Student Senate shall consist of all Senator positions that are currently occupied.
\item No voting member of the Student Senate shall simultaneously hold a position on the Rensselaer Union Judicial Board.
\end{enumerate}

\article{Officers}
\begin{enumerate}
\item The Grand Marshal is the presiding officer of the Senate.
\item All Officers of the Senate, with the exception of the Grand Marshal, shall be nominated by the Grand Marshal and confirmed
by a $\frac{2}{3}$ vote of the Senate.
\item The Grand Marshal assumes the duties of any unfilled officer position, with the exception of the Rules and Elections
Committee Chair.
\item It shall be the responsibility of the Vice Chair to:
\begin{enumerate}
\item Publicize meeting times.
\item Meet regularly with the Grand Marshal to discuss Senate issues.
\item Act as the substitute for the Grand Marshal at any meetings that the Grand Marshal cannot attend. At such a meeting the
Vice Chair shall not count towards quorum and may only vote to break a tie.
\item Become familiar with the work of all the Senate committees, and assist committees in their duties.
\item Assist individual Senators that have been charged with investigating issues raised by petition.
\item Coordinate annual documentation folders with committee chairmen to ensure continuity between years.
\end{enumerate}

\item It shall be the responsibility of the Secretary to:
\begin{enumerate}
\item Take meeting attendance and minutes.
\item Distribute minutes, committee reports, motions, and other printed material to the members of the Student Senate.
\end{enumerate}

\item The Treasurer shall be responsible for:
\begin{enumerate}
\item Managing funds allocated to the Student Senate as directed by the Grand Marshal or the Student Senate.
\item Preparing the annual budget of the Student Senate, subject to the approval of the Grand Marshal.
\item Approving transactions conducted by the Senate or any of its Committees. The Grand Marshal may overrule any decision
pertaining to such transactions.
\end{enumerate}

\item The Parliamentarian shall be responsible for:
\begin{enumerate}
\item Advising the Chair on matters of parliamentary procedure.
\item Maintaining an orderly meeting, abiding by rules of decorum.
\item Maintaining an accurate record of all legislation brought before the Senate.
\end{enumerate}

\item The Student Senate representative on the Executive Board shall be known as the Senate-Executive Board Liaison. They shall be
responsible for:
\begin{enumerate}
\item Expressing the Senate’s opinions to the Executive Board and the Executive Board’s opinions to the Senate.
\item Reviewing all legislation pertinent to Union fiscal policy and student fees.
\end{enumerate}
\item The Chair of the Rules and Elections Committee shall be an officer of the Senate. The Chair may not necessarily be a Senator,
but shall regularly attend Senate meetings. 
\end{enumerate}


\article{Duties of Office}
\begin{enumerate}
\item Undergraduate Senators shall be expected to either chair one Senate committee or sit on at least two Senate committees.
Graduate Senators shall be expected to either chair one Senate Committee or sit on at least one Senate committee.
\item Senators shall be responsible to dutifully attend all Senate Meetings, unless they obtain an excuse from the Grand Marshal.
Senators shall actively contribute to the Student Senate, and shall not hinder its performance.
\item Senators shall be expected to maintain satisfactory attendance in the council that they represent. Satisfactory attendance shall
be defined by each council’s Bylaws.
\item Failure to meet any of the duties of office shall subject the offending Senator to the removal process as outlined in Article IX of
these Bylaws.
\end{enumerate}


\article{Meetings}
\begin{enumerate}
\item The Student Senate shall meet at least twice a month during normal academic periods. The time and location of these meetings
shall be at the discretion of the Grand Marshal.
\item All meetings of the Student Senate shall be open to the public.
\item The Senate may, by a $\frac{2}{3}$ vote, close a meeting. Within a closed meeting the following guidelines must be met:
\begin{enumerate}
\item All motions, resolutions, and votes passed during the closed meeting must be made public immediately after the meeting is
opened.
\item The Senate may invite non-Senate members to remain by a majority vote.
\end{enumerate}
\item A quorum shall consist of $\frac{2}{3}$ of the voting membership of the Student Senate.
\item The Student Senate shall use the current edition of Robert’s Rules of Order as their parliamentary authority for any procedures
not specified in these Bylaws.
\item Meetings may be called by one-quarter of the voting membership of the Senate or by the Grand Marshal. The requesting
Senators may name a Senator other than the Grand Marshal to preside over the meeting. The Grand Marshal must be notified of
any meeting.
\begin{enumerate}
\item The presiding officer is responsible for informing all Senators at least twenty-four hours prior to the meeting as to the date,
time, place, and purpose of the meeting. Normal procedures for running this meeting must be followed.
\end{enumerate}
\end{enumerate}

\article{Legislation}
\begin{enumerate}
\item Legislation may be introduced for consideration by the Senate in any meeting through the following ways:
\begin{enumerate}
\item As a proposal, report, or recommendation of a standing committee or subcommittee.
\item As a proposal from the Graduate Council or the Undergraduate Council.
\item As a proposal from the Interfraternity Council or the Panhellenic Council.
\item As a proposal sponsored by two or more Senators.
\end{enumerate}
\item All proposed legislation must be submitted to the Grand Marshal who will, when appropriate, assign the legislation to the
applicable committee for study and recommendation. In some cases this will occur on the Senate floor to ensure proper
consideration within committee first and to allow business to proceed.
\item Legislation shall be submitted to the Grand Marshal no later than one week prior to the meeting during which it is to be
motioned and discussed. Legislation that is proposed by two or more Senators may be submitted and discussed within the same
meeting.
\item The Student Senate Bylaws shall not be suspended unless such an action has been approved by a $\frac{2}{3}$ vote at the meeting prior.
\item The Student Senate may not unilaterally repeal a by-law of another body.
\item Where not provided for in the Union Constitution, it shall require a majority vote of to approve a constitution or set of Bylaws
submitted to it by another body.
\item The Student Senate shall keep a record of all legislation that comes into consideration. To ensure accurate record keeping and to
provide context for posterity, the Senate shall include any supplementary materials that are relevant to the recorded legislation. \end{enumerate}

\article{Rules for Voting and Debate}
\begin{enumerate}
\item Upon entering debate, a queue will be established and maintained by the Parliamentarian.
\begin{enumerate}
\item The queue determines who has the floor to speak. The Parliamentarian is responsible for acknowledging all those who wish
to speak, and must announce each speaker before their speaking time begins.
\item The queue’s topic defers to the discussion at hand.
\item Total debate time is unbounded by default.
\item Speaking time defaults to five minutes.
\item Anyone present in the room has the right to speak, and may speak any number of times.
\begin{enumerate}
\item Those who have not spoken will be called upon before those who have spoken.
\item If the queue is on the topic of a presentation, designated presenters may respond to any questions directed at them.
Speaking time limits still apply.
\item The Parliamentarian may call any speaker off-topic or out of time.
\end{enumerate}

\item Queues for any motion will proceed until the motion has left the floor. Other queues may be ended by the presiding officer
when empty.
\item A Senator may move to restrict the queue, requiring a second and majority approval.
\begin{enumerate}
\item By default, only members of the Senate, Officers, and Committee Chairs may be added to a restricted queue.
Additional individuals to be given queue rights may be specified by the Senator moving to restrict the queue.
\item If an individual without queue rights wishes to speak during a restricted queue, they will not be called upon until no
members with queue rights remain on the queue.
\end{enumerate}

\item A Senator may move to close the queue, requiring a second and a $\frac{2}{3}$ vote.
\begin{enumerate}
\item No additional individuals may be added to a closed queue.
\end{enumerate}

\item A Senator may move to reopen a closed or restricted queue, requiring a second and a majority vote.
\item When speaking time is limited, speakers can yield their time in the following ways:
\begin{enumerate}
\item Yield to the Chair, forfeiting their time.
\item Yield to another speaker, transferring their time to another person in the room. That person may decline to speak,
forfeiting their time.
\item A yielded speaker may not yield time further.
\end{enumerate}

\item If unfriendly amendments or subsidiary motions requiring debate are made, the effective queue will be suspended and the
Parliamentarian will open a new queue.
\begin{enumerate}
\item When the amendment or subsidiary motion leaves the floor, its corresponding queue will be ended, and the original
queue will be resumed.
\end{enumerate}
\end{enumerate}

\item Senators are the only members of the Student Senate that have the right to vote on any business of the Senate.
\begin{enumerate}
\item The presiding officer may only vote to break a tie, unless otherwise specified within the Union Constitution, these Bylaws,
or Robert’s Rules of Order. The presiding officer is permitted to abstain. If a tie remains, the motion fails.
\end{enumerate}

\item All voting shall be by a show of hands, unless a Senator, Officer, Committee Chair, or the presiding officer requests a roll call.
\begin{enumerate}
\item In a show of hands, the presiding officer will first request the number in favor, then the number opposed. The presiding
officer will then announce the vote, and whether the motion has passed. In the event of a tie, the presiding officer may then
announce their vote followed by the outcome.
\item If a roll call vote is requested, the Secretary will read the names of the Senators alphabetically by surname. If the Grand
Marshal is included in the voting population, he or she will be called last.
\item Each Senator may vote “yes”, “no”, or “pass”. If they pass, their name will be read again after the first completion of the
roll call. At this time, the Senator must vote or abstain.
\item No vote may be determined, nor any result announced, until all Senators present have voted or abstained.
\end{enumerate}

\item All legislation considered by the Student Senate requires majority approval unless mentioned otherwise in these Bylaws, or in
Robert’s Rules of Order. All votes, unless specified within these Bylaws or in Robert’s Rules of Order, are debatable.
\end{enumerate}

\article{Removals}
\begin{enumerate}
\item Good cause for removal, as provided in the Rensselaer Union Constitution, shall be defined as follows:
\begin{enumerate}
\item Two or more unexcused absences from meetings of the Student Senate.
\item The failure of a member of the Student Senate to conscientiously fulfill the duties of office, as defined in Article V of these
Bylaws.
\end{enumerate}

\item Senators with more than three unexcused absences from a committee of which they are a member may be removed from the
committee. An absence may be excused by the Committee Chair, whose decision may be overruled by the Grand Marshal. 

\item No removal vote shall be taken unless an impeachment vote has been passed by a majority vote at the previous meeting. An
impeachment vote may be brought forth at the request of the organization which the Senator represents, or at the request of two
or more Senators.
\end{enumerate}

\article{Committees}
\begin{enumerate}

\item The Chair of each Student Senate standing committee except the Rules and Elections Committee shall be nominated by the
Grand Marshal and confirmed by a majority vote. Committee chairs serve at the pleasure of the Grand Marshal.
\begin{enumerate}
\item A Vice Chair may be selected by the Chair to assist in management and communication for the committee.
\item For each committee project or initiative, a Project Lead shall be selected by the chair.
\begin{enumerate}
\item Each Project Lead may create and lead a student task force, at the counsel of the Committee Chair, to handle initiatives
in their area requiring significant effort.
\end{enumerate}
\end{enumerate}

\item Committee membership shall be open to all members of the Rensselaer Union.
\begin{enumerate}
\item Any committee member who is not a member of the Senate must be approved by the Committee Chair before receiving
membership in the committee.
\item Members of a Committee serve at the pleasure of the Committee Chair and of the Grand Marshal, and may be removed at
any time for good cause.
\item Good cause for removal includes, but is not limited to, neglecting the duties of membership, failure to adhere to the rules of
order during meetings, or failure to adhere to these Bylaws when conducting business in the name of the committee.
\begin{enumerate}
\item Members are expected to dutifully attend meetings of the committee and complete committee assignments within reasonable deadlines.
\end{enumerate}
\end{enumerate}

\item All committees shall be empowered to create subcommittees as necessary to aid them in performing their functions. The
committee chair shall be an ex-officio member of any such created or existing subcommittee.

\item All committees shall meet twice per month during regular academic periods to work on projects, investigate pending
legislation, and to consider proposals for new legislation. Attendance shall be recorded at all committee meetings and submitted
to the Grand Marshal. All committee chairs shall report on any open projects within their committees’ purview at each Senate
meeting.

\item Each committee chair shall be held accountable for their committee’s performance.

\item During meetings, each Committee shall follow rules of order determined at the discretion of its Chair. Such rules of order shall
be made public immediately following their adoption and before any official committee business may commence.

\item The Rules and Elections Committee:
\begin{enumerate}
\item Consists of a maximum of ten members:
\begin{enumerate}
\item The Chair, one representative from each of the Graduate Council, Undergraduate Council, Executive Board, Judicial
Board, Interfraternity Council, Panhellenic Council, one Independent\footnote{Independent shall be defined as any student not affiliated with a Greek organization} member appointed by the Chair and approved
by a majority vote of the committee, one member-at-large appointed by the Chair and approved by a majority vote of
the committee, and one member-at-large from the voting membership of the Student Senate appointed by the Grand
Marshal.
\item Each of these groups shall appoint its own representative by the end of the second full week of the fall semester.
Should any of the memberships, with the exception of the chair, not be appointed by the required time or be vacant
for more than three weeks, the chair shall have the option of appointing a member-at-large. The group responsible for
appointing a person to that membership shall forfeit its right to do so until the following semester.
\item All members must be Activity Fee paying students. All members, with the exception of the chair, shall serve through
the end of the spring semester, until replaced, or until resigned, whichever comes first. The chair shall serve until they
resign or are replaced.
\item Quorum for the Rules and Elections Committee shall be $\frac{2}{3}$ of the total membership rounded up to the nearest
person. The Rules and Elections Committee shall not meet unless a Senator is appointed to the committee.
\item A member of the Rules and Elections Committee who is a candidate for Grand Marshal, President of the Union or
joins a political party or is a candidate assistant, as defined by the Student Senate or the Rules and Elections
Committee, will automatically be removed from the Rules and Elections Committee.
\end{enumerate}

\item Its duties shall include the following:
\begin{enumerate}
\item The Committee shall be responsible for reviewing and approving the language of all constitutional amendments that
come before the Senate for approval.
\item It shall keep a register of all Bylaws and constitutions of student organizations.
\item It shall investigate and report on the constitutional implications of all pending legislation. 
\item It shall be responsible for properly publicizing, conducting and supervising all elections for officers and
representatives of the governing bodies of the Union. It shall submit a report of the election results to the Student
Senate following the election.
\item It shall, at the discretion of the Senate, supervise or investigate any other campus election.
\item It shall, at the discretion of the Senate, supervise or investigate the student petition process.
\item It shall, at the discretion of the Senate investigate and report on any campus student government election.
\item It shall draft and submit to the Senate for approval rules for the conduct of the elections and their publicity.
\item It shall draft and submit to the Senate for approval rules governing the petition process.
\item It shall prepare and submit to the Judicial Board and Senate reports on all contested elections, which it investigates.
\end{enumerate}

\item Shall have the following standing subcommittee:
\begin{enumerate}
\item The Grand Marshal Week Committee shall be responsible for the planning and implementation of all activities that
are not directly connected with elections during Grand Marshal Week.
\end{enumerate}

\item The Chair shall appoint a vice chair from the voting membership of the committee. Duties of the Vice Chair shall include
assisting the Chair in the day-to-day administration of the committee. In the event that the Chair is absent or unable to run a
meeting the Vice Chair shall have the authority to chair a meeting. At such a meeting, quorum shall be $\frac{2}{3}$ the voting
membership, less the Vice Chair. The Vice Chair shall not have a vote at any meeting that they chair. In the event of the
resignation, removal, or extended absence of the Chair, the Vice Chair shall assume the duties of the Chair, until the Chair
is able to resume their duties, or until a new chair has been appointed.
\end{enumerate}

\item The Student Government Communications Committee shall be a standing committee of the Student Senate and the Executive
Board. It shall be responsible for promoting the initiatives and activities of the Student Senate and Executive Board.
\begin{enumerate}
\item Its Chair shall serve at the pleasure of the Grand Marshal and the President of the Union, and shall meet with the Grand
Marshal, the President of the Union, or their representatives at least twice a month to maintain cohesion and
communication between their respective bodies.
\item Its Chair may act as the spokesperson for the Senate and Executive Board at the discretion of the Grand Marshal and the
President of the Union, responsible for press releases and other communication with media outlets.
\item Its membership shall include at least one member of the Senate, at least one member of the Executive Board, and the Vice
Chairs of the following Senate Committees:
\begin{enumerate}
\item Academic Affairs Committee
\item Facilities and Services Committee
\item Hospitality Services Advisory Committee
\item Student Life Committee
\end{enumerate}
\item It shall make use of a variety of communication media to publicize the initiatives and activities of the Senate and the
Executive Board.
\begin{enumerate}
\item To this end, it shall coordinate with other Senate Committees to promote current projects and initiatives.
\item It shall provide the student body with accurate information on the activities, initiatives, and projects of the Senate.
\end{enumerate}
\item It shall make efforts to ascertain the concerns of students, and the opinions of students regarding matters under discussion
in the Senate and the Executive Board.
\begin{enumerate}
\item It may solicit student concerns and opinions through public relations initiatives.
\item It shall inform other Committees of student concerns pertaining to the purview of each Committee.
\end{enumerate}

\item It shall foster, among the student body, informed discussion of issues currently facing the Senate and the Executive Board.
\begin{enumerate}
\item It shall, when necessary, create and distribute informational media in conjunction with its ongoing public relations
initiatives.
\item It shall create and maintain platforms for student inquiry into the activities of the Senate and the Executive Board.
\item It may coordinate publication of electronic media with the Web Technologies Committee when necessary.
\item It shall attempt to create an open dialogue between the student body and the Institute Administration through meetings
and public relations events.
\end{enumerate}

\item It shall promote cohesion among members and Committees of the Senate and the Executive Board.
\begin{enumerate}
\item To this end, it shall promote and facilitate internal communications of the Senate and the Executive Board.
\end{enumerate}
\end{enumerate}

\item The Academic Affairs Committee shall be responsible for initiatives and legislation pertinent to academic curricula, academic
advising, post-graduate success, the Registrar’s office, and student-faculty relations.
\begin{enumerate}
\item It shall assess the state of Rensselaer academic courses.
\begin{enumerate}
\item To this end, it shall coordinate with the Student Government Communications Committee during outreach programs
to ascertain student opinions and concerns regarding academics.
\item It shall work with relevant organizations within the Institute to identify areas for improvement regarding academics.
\end{enumerate}
\item It shall attempt to maintain a diverse membership that represents various academic departments.
\begin{enumerate}
\item It shall attempt to include graduate and undergraduate representation on the committee.
\end{enumerate}
\item It shall inform the Senate and the student body of any changes in academics at the Institute.
\item It shall be responsible for selecting the undergraduate representatives to the Faculty Senate Curriculum Committee.
\item It shall be responsible for maintaining regular contact between the Student Senate and the Faculty Senate.
\end{enumerate}

\item The Union Annual Report Committee shall be a standing committee of the Student Senate and Union Executive Board.
\begin{enumerate}
\item It shall be chaired by the Senate-Executive Board Liaison.
\item It shall work to produce the Union Annual Report, which will detail the Union Activity Fee for both graduate and
undergraduate students; this report will be presented to the Student Senate for approval.
\item Its membership shall consist of the Director of the Union, as a nonvoting member; the Business Administrator, as a
nonvoting member; the President of the Union as a non-voting ex-officio member; at least one graduate member of the
Student Body; and at least one undergraduate member of the Student Body.
\end{enumerate}

\item The Facilities and Services Committee shall be responsible for initiatives and legislation pertinent to the facilities and services
of the Institute, notably those under the Administration Division.
\begin{enumerate}
\item It shall take steps to gather information on the state of services and facilities.
\begin{enumerate}
\item To this end, it shall coordinate with the Student Government Communications Committee during outreach programs to
ascertain student opinions and concerns on the state of services and facilities.
\item It shall, when necessary, address these concerns and coordinate its efforts with proper Institute personnel.
\item It shall, from time to time, inspect the overall state of facilities and services and report its findings to the Senate.
\end{enumerate}

\item It shall take steps to remedy student concerns regarding facilities and services.
\begin{enumerate}
\item To this end, it shall be responsible for contacting proper Institute personnel to discuss practical solutions to issues
regarding facilities and services.
\item It shall attempt to implement these solutions, with the support of the Institute and the Senate.
\end{enumerate}
\end{enumerate}

\item The Hospitality Services Advisory Committee shall be a standing subcommittee of the Facilities and Services Committee. It
shall be responsible for initiatives and legislation pertinent to the quality and future direction of dining services on the RPI
campus.
\begin{enumerate}
\item It shall take steps to gather information on the state of Rensselaer dining services.
\begin{enumerate}
\item It shall develop, maintain, and advertise tools and methods for students to deliver feedback and input into the quality of
campus food, dining halls, and other dining services.
\item It shall, from time to time, inspect the overall state of dining halls and programs and report its findings to the Senate.
\end{enumerate}
\item It shall take steps to remedy student concerns regarding campus dining.
\begin{enumerate}
\item It shall meet with Hospitality Services on a regular basis, at a time and place coordinated between the Committee Chair
and relevant administrators.
\item It shall pursue, research, and propose dining programs that align with student needs.
\item It shall collaborate with Hospitality Services and other organizations to initiate solutions to student concerns regarding
Rensselaer dining services.
\end{enumerate}

\item It shall attempt to maintain a diverse membership that represents a wide range of demographics with regards to campus
dining.
\end{enumerate}

\item The Student Life Committee shall be responsible for initiatives and legislation pertinent to student rights, residence life,
quality of life and well-being, and the Office of Student Life portfolio.
\begin{enumerate}
\item It shall take steps to gather information on the state of residence life, quality of life, and well-being.
\begin{enumerate}
\item To this end, it shall coordinate with the Student Government Communications Committee during outreach programs to
ascertain student opinions and concerns within these areas.
\item It shall, from time to time, inspect the state of the Residence Halls and report its findings to the Senate.
\end{enumerate}
\item It shall address issues pertaining to student rights and responsibilities as they arise.
\begin{enumerate}
\item It shall work with the Dean of Students to discuss pending changes to the Handbook of Student Rights and
Responsibilities, or other relevant policies that affect student rights.
\end{enumerate}
\item It shall take steps to remedy student concerns regarding student life.
\begin{enumerate}
\item It shall advise the administration on issues related to student life, and work with them in addressing these issues.
\item It shall work and collaborate with the administration and other organizations to initiate new programs that improve the
student experience.
\end{enumerate}
\item It shall sustain and improve relations between the community at large and Rensselaer.
\begin{enumerate}
\item It shall provide students with opportunities to interact with the Troy community.
\end{enumerate}
\end{enumerate}

\item The Web Technologies Group shall be responsible for promoting, implementing, and maintaining technological initiatives of
the Student Senate.
\begin{enumerate}
\item Its chair shall be responsible for assisting the Grand Marshal and Senate Committees in addressing their technological
needs.
\item It shall design and implement technological solutions to student concerns, collaborating with other committees and
organizations as necessary.
\item It shall contribute to Rensselaer Union homepage development efforts, and shall be responsible for maintenance and
updates of all other Student Government web tools and services.
\item Shall be responsible for historical cataloging of documents and information for a digital Student Government archive.
\end{enumerate}
\end{enumerate}


\article{Petitions}
\begin{enumerate}
\item Petitions shall be a means by which members of the Union may bring concerns, initiatives, or ideas before the Student Senate.
\begin{enumerate}
\item Petitions may only be created and signed by members of the Union.
\begin{enumerate}
\item Should a signatory lose their membership status during the duration of a petition they have signed, their signature on that
petition shall remain valid.
\end{enumerate}
\item The Rules and Elections Committee shall maintain rules governing the petition process.
\begin{enumerate}
\item Such rules shall be approved by a majority vote of the Senate.
\end{enumerate}
\end{enumerate}

\item Should any petition, compliant with rules governing the petition process, reach a threshold of 250 signatures within one full
calendar year of its posting, it shall be placed on the agenda at a Senate meeting occurring within the next fifteen academic
days, excluding exam periods.
\begin{enumerate}
\item The Senate may choose to address petitions at any time before this.
\item Should the Senate’s term end within fifteen academic days of a petition reaching this threshold, that petition may instead be
addressed by the next Senate in the semester of their election.
\end{enumerate}

\item The Senate may choose to take the following courses of action when addressing a petition:
\begin{enumerate}
\item Charge one or more Senators to investigate issues raised by the petition.
\item Charge a committee to investigate the petition as an initiative.
\item Vote on a resolution on the issue.
\item Vote to hold a referendum election on the issue.
\item Refer a petition to another organization as appropriate.
\end{enumerate}
\end{enumerate}

\article{Appointments}
\begin{enumerate}
\item The Grand Marshal shall make any appointments not specified by these or the Bylaws of other respective boards.
\item Appointments made by the Grand Marshal must be reported to the Senate at its next scheduled meeting, or will be automatically nullified.
\end{enumerate}


\article{Amendments}
\begin{enumerate}
\item The procedure to amend the Bylaws shall be as follows:
\begin{enumerate}
\item All proposed amendments to the Student Senate Bylaws must be proposed in their entirety and passed by a $\frac{2}{3}$ vote of the
Senate.
\item The Rules and Elections Committee Chair shall advise the Senate on the consistency of any proposed amendment with the
Union Constitution and any other governing documents.
\end{enumerate}

\item The Senate Bylaws shall be amended as necessary so as not to be in conflict with outside laws, the Rensselaer Polytechnic
Institute Corporate Charter, the Rensselaer Polytechnic Institute Bylaws, the Rensselaer Polytechnic Institute Student Bill of
Rights, the Rensselaer Union Constitution, and decisions of the Board of Trustees of Rensselaer Polytechnic Institute.

\item These Bylaws shall be modified to reflect an approved amendment by either:
\begin{enumerate}
\item Inserting new clauses into these Bylaws at their appropriate places, or
\item Removing clauses from these Bylaws.
\end{enumerate}
\end{enumerate}

\article{Succession}
\begin{enumerate}
\item If the Grand Marshal is removed from office, has an extended leave of absence, or is otherwise unable to fulfill the duties or
requirements of office, then the Vice Chair shall assume the duties of the Grand Marshal as Acting Chair of the Senate.
\item The Acting Chair of the Senate assumes the duties of Grand Marshal until such a time that a meeting of the Senate can be
called. The first and only order of business of such a meeting shall be the selection of a new Grand Marshal.
\item All nominations for the position of Grand Marshal must be motioned and seconded by the members of the Senate. The
nomination processes shall be overseen by the Chair of the Judicial Board.
\begin{enumerate}
\item All nominees must be Senators as defined in these Bylaws.
\end{enumerate}
\item In the case of multiple nominees, an appointment for Grand Marshal shall be selected by a series of simple majority votes.
\begin{enumerate}
\item If there are two nominees, a vote shall be taken, and the nominee with a simple majority shall be named the appointment
for Grand Marshal.
\item If there are three or more nominees, a series of votes shall be taken. 
\begin{enumerate}
\item In each vote, the nominee with the least number of votes shall be removed from consideration, and only the remaining
nominees shall be considered in the next vote.
\item This series of votes shall continue until a single nominee remains, and that nominee shall be named the appointment for
Grand Marshal.
\end{enumerate}
\end{enumerate}

\item The nominee who is named the appointment for Grand Marshal must be approved by $\frac{2}{3}$ of the Senate’s total voting
membership. If the appointment is not approved, a new process of nominations begins.
\item Once a new Grand Marshal has been named, the Acting Chair of the Senate shall relinquish the responsibilities of office to the
new Grand Marshal.
\item The new Grand Marshal may appoint new officers of the Senate and committee chairs.

\end{enumerate}

\end{document}